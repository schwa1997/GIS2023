\section{2. Analysis of Needs}

\subsection{2.1 Adopted Methods}
For the analysis of needs in our vineyard survey project, we have adopted a combination of field surveys, stakeholder consultations, and technological assessment. Field surveys involve physically visiting vineyard sites to assess their conditions, disease prevalence, and existing management practices. Stakeholder consultations include engaging with agronomists, vineyard managers, and winery owners to gather insights into their requirements and pain points. Technological assessment involves evaluating existing software tools, geographic information systems (GIS), and remote sensing technologies suitable for disease detection and management in vineyards.

\subsection{2.2 Description of the Identified Needs}
Through our adopted methods, we have identified several critical needs for our vineyard survey project. These include:
\begin{itemize}
    \item Accurate Disease Detection: The need for a reliable system to detect vine diseases early and accurately to prevent significant crop losses.
    \item Data Integration: The need to integrate various data sources, such as geospatial information, historical disease data, and intervention records.
    \item Decision Support: The need for a platform that offers actionable insights and recommendations to agronomists for effective disease management strategies.
    \item Collaboration: The need for a collaborative environment where agronomists, vineyard managers, and other stakeholders can share data and insights.
    \item Scalability: The need for a solution that can accommodate vineyards of varying sizes and geographical locations.
\end{itemize}

\subsection{2.3 Description of Data Needs}
In our vineyard survey project, the following types of data are crucial:
\begin{itemize}
    \item Geospatial Data: Accurate geographic information about vineyard locations, boundaries, and terrain for mapping and analysis.
    \item Disease Data: Historical and real-time data on disease occurrences, types, and severity to track trends and assess risks.
    \item Intervention Records: Information about previous intervention strategies, their outcomes, and effectiveness.
    \item Weather Data: Local weather conditions and forecasts to correlate with disease outbreaks and plan interventions.
    \item Soil Information: Soil composition, moisture levels, and nutrient content for informed decision-making.
\end{itemize}